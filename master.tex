%~ Version 1.5 of the PDF specification
\pdfminorversion=5
%~ compression of non-stream objects
\pdfobjcompresslevel=2

\documentclass[
    t,
    smaller,
    compress,
    %~ draft
]{beamer}
\usepackage[utf8]{inputenc}
\usepackage[ngerman]{babel}
\usetheme{BerlinFU}



%~ \usefonttheme{professionalfonts}
%~ \usepackage[noae]{Sweave}


\hypersetup{pdfstartview={FitH}} % By default the pdf fit the width of the screen.
%~ \hypersetup{pdfpagemode=FullScreen}


% http://tex.stackexchange.com/questions/16357/how-can-i-position-an-image-in-an-arbitrary-position-in-beamer
\usepackage{tikz}
\tikzset{
  every overlay node/.style={
    anchor=north west,
  },
}
% Usage:
% \tikzoverlay at (-1cm,-5cm) {content};
% or
% \tikzoverlay[text width=5cm] at (-1cm,-5cm) {content};
\def\tikzoverlay{%
   \tikz[baseline,overlay]\node[every overlay node]
}%


% http://tex.stackexchange.com/questions/48473/best-way-to-give-sources-of-images-used-in-a-beamer-presentation
\usepackage[absolute,overlay]{textpos}

\setbeamercolor{framesource}{fg=gray}
\setbeamerfont{framesource}{size=\footnotesize}

\newcommand{\source}[1]{
    \vspace{-4pt}
    \begin{beamercolorbox}[ht=4pt,right]{framesource}
        \usebeamerfont{framesource}\usebeamercolor[fg]{framesource} Quelle: {#1}
    \end{beamercolorbox}
}


\defbeamertemplate*{section in toc}{my theme}
{\leavevmode\leftskip=0em\large{\usebeamercolor[fg]{titlelike}\inserttocsectionnumber.} \inserttocsection\par\vspace{10pt}}

\defbeamertemplate*{subsection in toc}{my theme}
{\leavevmode\leftskip=2em\normalsize{\usebeamercolor[fg]{titlelike}\inserttocsectionnumber.\inserttocsubsectionnumber.} \inserttocsubsection\par}

\defbeamertemplate*{subsubsection in toc}{my theme}
{\leavevmode\leftskip=3.5em\normalsize\usebeamerfont{subsection in toc}\usebeamerfont{subsubsection in toc}\inserttocsubsubsection\par}


%%% Formatting table %%%%%%%%%%%%%%%%%%%%%%%%%%%%%%%%%%%%%%%%%%%%%%%%%%%%%%%%%%%%
%\usepackage{booktabs, colortbl, tabularx}


%% PLOT %%%%%%%%%%%%%%%%%%%%%%%%%%%%%%%%%%%%%%%%%%%%%%%%%%%%%%%%%%%%%%%%%%%%%%%%
\usetikzlibrary{shapes, intersections, calc, backgrounds}
%% Ploting
\usepackage{pgfplots}
\usepackage{pgfplotstable}


%~ Anzahl Folien: pro Folie solltet ihr mindestens 90s einplanen
%~
%~ Übergänge roter Faden


%~ \begin{frame}
  %~ \frametitle{}
  %~ \begin{columns}[T]
    %~ \begin{column}{.5\textwidth}<1->
      %~
    %~ \end{column}
    %~ \begin{column}{.5\textwidth}<1->
      %~ \begin{figure}
        %~ \includegraphics[height=40mm]{}
        %~ \source{[\ref{}]}
      %~ \end{figure}
    %~ \end{column}
  %~ \end{columns}
%~ \end{frame}


\title{Webframeworks für Python}
\subtitle{Vorstellung und Vergleich}
\author{Andreas Knöpfle, Bastien Sachs und Tobias Schmid}
\institute{Institut für Informatik}
\date{28.\,November 2012}
%~ \titlegraphic{silberlaube2}


\begin{document}


{
\setbeamertemplate{footline}{}
\begin{frame} % Titelseite
    \titlepage
\end{frame}
}


\begin{frame}
  %\frametitle{}
  Einstiegs-Idee:\\
  viele Logos der Webframeworks für Python\\
  Stichpunkte:\\
   - In Python geschriebene Webframework gibt es wie „Sand am Meer“\\
   - Warum? Einblick gibt unser Vortrag\\
\end{frame}


\begin{frame}
  \frametitle{Gliederung}
  \tableofcontents
  %~ roter Faden
\end{frame}


\section{Python~-- Programmiersprache fürs Web?}
\begin{frame}
  \frametitle{Python~-- Programmiersprache fürs Web?}

  \begin{itemize}[<1->]
    \item 1991 erstmals erschienen, aktuelle Versionen: 3.3.0 (September 2012), 2.7.3 (April 2012)
    \item Entwurfsphilosophie betont Programmlesbarkeit
    \begin{itemize}[<1->]
      \item Blöcke durch gleiche Einrückung begrenzt
      \item relativ wenige Schlüsselwörter
    \end{itemize}
    %\item Dynamische Programmiersprache
   % Python-Interpreter mit interaktiven Modus
    \item objektorientierte, aspektorientierte und funktionale Programmierung
    \item dynamische Datentypen, garbage collection
    \item große, umfangreiche Standardbibliothek “batteries included”
  \end{itemize}
  %~  * Plugins für Eclipse, NetBeans, Vim, Emacs und v.a.

  Logo Python
\end{frame}


\begin{frame}
  \frametitle{Python~-- Programmiersprache fürs Web?}

  eventl. Codebeispiele, Gegenüberstellungen, Erklärung

\end{frame}


\begin{frame}
  \frametitle{Web Server Gateway Interface (WSGI)}
 * Apache (mod\_wsgi), nginx (uWSGI)

 Bild
 % http://de.wikipedia.org/wiki/Web_Server_Gateway_Interface
\end{frame}


\section{Webframeworks für Python}
\begin{frame}
  \frametitle{Unterteilung Webframeworks für Python}
  Popular Full-Stack Frameworks, Other Full-Stack Frameworks, Basic Frameworks
 * Django\\
 * CherryPy, CubicWeb, Flask, Grok, Plone, Pylons, Pyramid, TurboGears, web2py, Zope 2\\
 * Bottle, Karrigell, Nagare, Pyjamas, Quixote, Spyce, Tornado, TwistedWeb, Web.py\\
 * Nicht mehr aktiv: BlueBream, Nevow, Webware\\
\end{frame}


\begin{frame}
Hauptteil:\\
Vor- und Nachteile einiger weniger Frameworks aufgezeigen\\
(konkrete) Lösungsansätze für bestimmte Probleme/Vergleichskriterien
\end{frame}


\begin{frame}
  \frametitle{Persistenzschicht}
\end{frame}


\begin{frame}
  \frametitle{Templates, I18N, L10N}
\end{frame}


\begin{frame}
  \frametitle{Konfiguration, Routing}
\end{frame}


\begin{frame}
  \frametitle{Formulare, Validierung}
  
  \begin{itemize}[<1->]
    \item django.forms
	 \begin{itemize}[<1->]
		\item HTML form widget
		\item Field validation
		\item ...
	\end{itemize}
  \end{itemize}

\end{frame}


\begin{frame}
  \frametitle{Sicherheitmechanismen}
  
  \begin{itemize}[<1->]
    \item django: http://www.djangobook.com/en/2.0/chapter20.html
    \end{itemize}

\end{frame}


\begin{frame}
  \frametitle{Bootstrapping, Scaffolding, Erweiterbarkeit}
\end{frame}


\begin{frame}
  \frametitle{Extras: WebServices, Caching, Tests}
\end{frame}


\begin{frame}
  \frametitle{Kriterienübersicht}
 * Vergleichstabellen (Django vs. ...)
\end{frame}


\section{Fazit}
\begin{frame}
  \frametitle{Fazit}

  Je Anforderungen an das Webframework (“Taste”)\\
  \dots
\end{frame}


\begin{frame}
  \frametitle{Quellen der Abbildungen}
  \footnotesize
  \begin{enumerate}[<1->]
    \item Innenhof Informatik
        \url{http://www.flickr.com/photos/bennybenny/3597853896/} \label{illu:1}
  \end{enumerate}
  alle URLs aufgerufen am 14. November 2012.
\end{frame}

\begin{frame}
  \frametitle{Quellen}
http://www.infoworld.com/d/application-development/pillars-python-six-python-web-frameworks-compared-169442
http://wiki.python.org/moin/WebFrameworks
http://wiki.python-forum.de/Web-Frameworks
http://blog.ianbicking.org/turbogears-and-pylons.html
\end{frame}


\begin{frame}
    \frametitle{Ende der Präsentation}
    \LARGE
    \begin{itemize}[<1->]
        \item Vielen Dank für Ihre Aufmerksamkeit.
        \item
        \item offene Fragen?
        \item Diskussion
        \begin{itemize}[<1->]
          \Large
          \item Kritik, Anregungen
        \end{itemize}
    \end{itemize}
\end{frame}

\end{document}
