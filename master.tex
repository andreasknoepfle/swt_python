\pdfminorversion=5      %~ Version 1.5 of the PDF specification
\pdfobjcompresslevel=2  %~ compression of non-stream objects

\documentclass[
    t,
    smaller,
    compress,
    %~ draft
]{beamer}
\usepackage[utf8]{inputenc}
\usepackage[ngerman]{babel}

\usepackage{color} % für Farben im allgemeinen
\usepackage{colortbl}% für die Hintergrundfarbe einzelner Zellen in Tabellen
\usepackage{listings,bera}
\definecolor{keywords}{RGB}{255,0,90}
\definecolor{dkgreen}{rgb}{0,0.6,0}
\definecolor{mauve}{rgb}{0.58,0,0.82}
\definecolor{backgroundc}{rgb}{0.95,0.95,0.95}
\lstset{language=Python,
basicstyle=\footnotesize, 
keywordstyle=\color{keywords},
commentstyle=\color{dkgreen},
 stringstyle=\color{mauve},
 breakatwhitespace=true,
 backgroundcolor=\color{backgroundc},
 frame=single, 
emph}

\usetheme{BerlinFU}


% --- Farbdefinitionen ----------------------------------------
%\definecolor{red}{rgb}{1,0,0}
%\definecolor{orange}{rgb}{0.5,0.5,0}
%\definecolor{green}{rgb}{0,1,0}

%~ \usefonttheme{professionalfonts}
%~ \usepackage[noae]{Sweave}


\hypersetup{pdfstartview={FitH}} % By default the pdf fit the width of the screen.
%~ \hypersetup{pdfpagemode=FullScreen}


% http://tex.stackexchange.com/questions/16357/how-can-i-position-an-image-in-an-arbitrary-position-in-beamer
\usepackage{tikz}
\tikzset{
  every overlay node/.style={
    anchor=north west,
  },
}
% Usage:
% \tikzoverlay at (-1cm,-5cm) {content};
% or
% \tikzoverlay[text width=5cm] at (-1cm,-5cm) {content};
\def\tikzoverlay{%
   \tikz[baseline,overlay]\node[every overlay node]
}%


% http://tex.stackexchange.com/questions/48473/best-way-to-give-sources-of-images-used-in-a-beamer-presentation
\usepackage[absolute,overlay]{textpos}

\setbeamercolor{framesource}{fg=gray}
\setbeamerfont{framesource}{size=\footnotesize}

\newcommand{\source}[1]{
    \vspace{-4pt}
    \begin{beamercolorbox}[ht=4pt,right]{framesource}
        \usebeamerfont{framesource}\usebeamercolor[fg]{framesource} Quelle: {#1}
    \end{beamercolorbox}
}


\defbeamertemplate*{section in toc}{my theme}
{\leavevmode\leftskip=0em\large{\usebeamercolor[fg]{titlelike}\inserttocsectionnumber.} \inserttocsection\par\vspace{10pt}}

\defbeamertemplate*{subsection in toc}{my theme}
{\leavevmode\leftskip=2em\normalsize{\usebeamercolor[fg]{titlelike}\inserttocsectionnumber.\inserttocsubsectionnumber.} \inserttocsubsection\par}

\defbeamertemplate*{subsubsection in toc}{my theme}
{\leavevmode\leftskip=3.5em\normalsize\usebeamerfont{subsection in toc}\usebeamerfont{subsubsection in toc}\inserttocsubsubsection\par}


%%% Formatting table %%%%%%%%%%%%%%%%%%%%%%%%%%%%%%%%%%%%%%%%%%%%%%%%%%%%%%%%%%%
%\usepackage{booktabs, colortbl, tabularx}


%% PLOT %%%%%%%%%%%%%%%%%%%%%%%%%%%%%%%%%%%%%%%%%%%%%%%%%%%%%%%%%%%%%%%%%%%%%%%%
%\usetikzlibrary{shapes, intersections, calc, backgrounds}
%% Ploting
%\usepackage{pgfplots}
%\usepackage{pgfplotstable}


%~ Anzahl Folien: pro Folie solltet ihr mindestens 90s einplanen
%~
%~ Übergänge roter Faden


%~ \begin{frame}
  %~ \frametitle{}
  %~ \begin{columns}[T]
    %~ \begin{column}{.5\textwidth}<1->
      %~
    %~ \end{column}
    %~ \begin{column}{.5\textwidth}<1->
      %~ \begin{figure}
        %~ \includegraphics[height=40mm]{}
        %~ \source{[\ref{}]}
      %~ \end{figure}
    %~ \end{column}
  %~ \end{columns}
%~ \end{frame}


\newlength\yearposx


\title{Webframeworks für Python}
\subtitle{Vorstellung und Vergleich}
\author{Andreas Knöpfle, Bastien Sachs und Tobias Schmid}
\institute{Institut für Informatik}
\date{28.\,November 2012}
%~ \titlegraphic{silberlaube2}


\begin{document}


{
\setbeamertemplate{footline}{}
\begin{frame} % Titelseite
    \titlepage
\end{frame}
}


\begin{frame}
  %\frametitle{}
  Einstiegs-Idee:\\
  viele Logos der Webframeworks für Python\\
  Stichpunkte:\\
   - In Python geschriebene Webframework gibt es wie „Sand am Meer“\\
   - Warum? Einblick gibt unser Vortrag\\
\end{frame}


\begin{frame}
  \frametitle{Gliederung}
  \tableofcontents
  %~ roter Faden
\end{frame}

\begin{frame}
	
	\frametitle{Farbdeutung bei Vergleichen}
	
	\begin{table}[h]
		\begin{tabular}{|c|}
			\hline
			 \cellcolor{dkgreen} sehr gute Lösung  \\ \hline
		  	\cellcolor{orange} in manchen Fällen evtl. nicht optimal \\ \hline 
		  	\cellcolor{red} nicht optimale Lösung \\ \hline 
		 \end{tabular}
	\end{table}
	
\end{frame}


\section{Python~-- Programmiersprache fürs Web?}
\begin{frame}
  \frametitle{Python~-- Programmiersprache fürs Web?}

  % 1991 erstmals erschienen, aktuelle Versionen: 3.3.0 (September 2012), 2.7.3 (April 2012)
  % http://en.wikipedia.org/wiki/History_of_Python
  \vspace{12pt}
  \begin{tikzpicture}[scale=0.48] % timeline 1991-2012->
    % define coordinates (begin, used, end, arrow)
    \foreach \x in {1991,1994,2004,2008,2012,2013}{
        \pgfmathsetlength\yearposx{(\x-1991)*1cm};
        \coordinate (y\x)   at (\yearposx,0);
        \coordinate (y\x t) at (\yearposx,+3pt);
        \coordinate (y\x b) at (\yearposx,-3pt);
    }

    % draw horizontal line with arrow
    \draw [->] (y1991) -- (y2013);

    % draw ticks
    \foreach \x in {1991,1994,2004,2008,2012}
        \draw (y\x t) -- (y\x b);

    % annotate (draw nodes)
    \foreach \x in {1991,1994,2004,2008,2012}
        \node at (y\x) [below=3pt] {\x};

    \node at (6.5cm, 2.5cm) {\pgftext{\includegraphics[height=1cm]{python_1990.png}}};
    \node at (18cm, 2.5cm) {\pgftext{\includegraphics[height=1cm]{python_logo_2006.png}}};

    \fill (y1991) circle (3pt) node[above=3pt] {0.9};
    \fill (y1994) circle (3pt) node[above=3pt] {1.0};
    \fill (y2004) circle (3pt) node[above=3pt] {2.0};
    \fill (y2008) circle (3pt) node[above=3pt] {3.0};
    \fill (y2012) circle (3pt) node[above=3pt] {2.7.3, 3.3.0};
  \end{tikzpicture}
  \vspace{12pt}

  \begin{itemize}[<1->]
    \item Entwurfsphilosophie betont Programmlesbarkeit
    \begin{itemize}[<1->]
      \item Blöcke durch gleiche Einrückung begrenzt
      \item relativ wenige Schlüsselwörter
    \end{itemize}
    %\item Dynamische Programmiersprache
   % Python-Interpreter mit interaktiven Modus
    \item objektorientierte, aspektorientierte und funktionale Programmierung
    \item dynamische Datentypen, garbage collection
    \item große, umfangreiche Standardbibliothek “batteries included”
  \end{itemize}

 %~ http://brainsik.theory.org/.:./2009/why-reddit-uses-python
 %~ http://www.quora.com/Quora-Infrastructure/Why-did-Quora-choose-Python-for-its-development

  %~  * Plugins für Eclipse, NetBeans, Vim, Emacs und v.a.
\end{frame}


\begin{frame}
  \frametitle{Python~-- Programmiersprache fürs Web?}

  eventl. Codebeispiele, Gegenüberstellungen, Erklärung

\end{frame}


\begin{frame}
  \frametitle{Web Server Gateway Interface (WSGI)}
 * Apache (mod\_wsgi), nginx (uWSGI), Gunicorn

 Bild
 % http://de.wikipedia.org/wiki/Web_Server_Gateway_Interface
\end{frame}


\begin{frame}
  \frametitle{Python~-- Programmiersprache fürs Web!}

  Zusammenfassung der genannten Punkte

\end{frame}


\section{Webframeworks für Python}
\begin{frame}
  \frametitle{Webframeworks für Python}
  	 Full-Stack Frameworks
  		\begin{itemize}
  			\item Django
  			\item TurboGears 
  			\item web2py
  			\item Pylons/Pyramid
  		\end{itemize}
  		 Microframeworks
  		\begin{itemize}
  			\item Bottle
  			\item CherryPy
  		\end{itemize}
  
% , Other Full-Stack Frameworks, Basic Frameworks
% * Django\\
% * CherryPy, CubicWeb, Flask, Grok, Plone, Pylons, %Pyramid, TurboGears, web2py, Zope 2\\
% * Bottle, Karrigell, Nagare, Pyjamas, Quixote, Spyce, %Tornado, TwistedWeb, Web.py\\
% * Nicht mehr aktiv: BlueBream, Nevow, Webware\\
\end{frame}


\begin{frame}
Hauptteil:\\
Vor- und Nachteile einiger weniger Frameworks aufgezeigen\\
(konkrete) Lösungsansätze für bestimmte Probleme/Vergleichskriterien
\end{frame}


\begin{frame}
  \frametitle{Persistenz}
  \begin{itemize}[<1->]
    \item SQL ORM
    \begin{itemize}
        \item Django built-in ORM (Django)
        \item SQLAlchemy (Grok,Pylons,Zope)
        \item Storm (Grok )
        \item SQLObject (Pylons,TurboGears)
        \item DAL (web2py)
     \end{itemize}
    \item MongoDB ORM
    \begin{itemize}[<1->]
        \item Django MongoDB Engine (Django)
        \item Ming (TurboGears)
    \end{itemize}
    \item ...
  \end{itemize}

\end{frame}


\begin{frame}
  \frametitle{Templates, I18N, L10N}
  \begin{itemize}[<1->]
    \item Template Engines:
        \begin{itemize}[<1->]
            \item Python built-in
            \item Django template language
            \item Cheetah (Django, Turbogears, Pylons)
            \item Myghty
        \end{itemize}
  \item I18N + L10N
  \begin{itemize}[<1->]
    \item gettext
   \end{itemize}
  \end{itemize}

\end{frame}


\begin{frame}
  \frametitle{Konfiguration, Routing}
	Django
  	\begin{itemize}[<1->]
    	\item URLconf (URL Konfiguration)
    	\item einfaches Mapping zwischen URL-Patterns (Regex)
    	\item Wenn der Ausdruck passt, ruft Django den View auf
    	\item settings.py für alle weiteren Konfigurationen
    	\begin{itemize}[<1->]
    		\item Datenbankeinstellungen
    		\item E-Mail und Fehlermeldungseinstellungen
    		\item i18n und URL Einstellungen
    		\item Applikations and Middleware Einstellungen
    	\end{itemize}
  	 \end{itemize}
  	 
\end{frame}


\begin{frame}
  \frametitle{Konfiguration, Routing}
TurboGears

"...extremely flexible for power users and very simple to use for standard projects."

  	 \begin{itemize}[<1->]
    	\item Object Dispatch, und built in Routes Integration
    	\item kann überschrieben werden
    	\item development.ini, test.ini, production.ini
    	
 	\end{itemize}
\end{frame}  
\begin{frame}[fragile]
\frametitle{Routing Beispiel Django}
\begin{lstlisting}
urlpatterns = patterns('',
	(r'^articles/2003/$','news.views.special_case_2003'),
	(r'^articles/(\d{4})/$','news.views.year_archive'),
	(r'^articles/(\d{4})/(\d{2})/$','news.views.month_archive'),
	(r'^articles/(\d{4})/(\d{2})/(\d+)/$','news.views.article_detail'),
)
\end{lstlisting}
	Requests:
	\begin{lstlisting}
		/articles/2005/03/		=> month_archive(request,'2005','03')
		/articles/2005/3/		=> no match
		/articles/2003/			=> special_case_2003(request,'2003')
		/articles/2003			=> no match
		/articles/2003/03/03/	=> article_detail(request,'2003','03','03')
	\end{lstlisting}
\end{frame}
\begin{frame}[fragile]
\frametitle{Routing Beispiel TurboGear 2.0}
\begin{lstlisting}	
		def setup_routes(self):
		
		map = Mapper(directory=config['pylons.paths']['controllers'],
		            always_scan=config['debug'])
		
		# Setup a default route for the root of object dispatch
		map.connect('*url',controller='root', action='routes_placeholder')

		config['routes.map'] = map
\end{lstlisting}
\end{frame}

\begin{frame}
	\frametitle{Vergleich Konfiguration, Routing}
	
	
	\begin{table}[h]
		\begin{tabular}{|c|c|c|}
			\hline
			 & Django & TurboGear 2.0  \\ \hline
		  	Konfiguration & \cellcolor{orange} mächtig, evtl. Overload  &  \\ \hline
		  	Routing & \cellcolor{orange} mächtig, evtl. Overload & \cellcolor{green} einfach,anpassbar	   \\ \hline
		 \end{tabular}
	\end{table}

	
\end{frame}

\begin{frame}
  \frametitle{Formulare, Validierung}
  
  \begin{itemize}[<1->]
    \item django.forms
	 \begin{itemize}[<1->]
		\item HTML form widget
		\item Field validation
		\item ...
	\end{itemize}
  \end{itemize}

\end{frame}


\begin{frame}
  \frametitle{Sicherheitmechanismen}
  
  \begin{itemize}[<1->]
    \item django: http://www.djangobook.com/en/2.0/chapter20.html
    \end{itemize}

\end{frame}


\begin{frame}
  \frametitle{Bootstrapping, Scaffolding, Erweiterbarkeit}
\end{frame}


\begin{frame}
  \frametitle{Extras: WebServices, Caching, Tests}
\end{frame}


\begin{frame}
  \frametitle{Kriterienübersicht}
 * Vergleichstabellen (Django vs. ...)
\end{frame}


\section{Fazit}
\begin{frame}
  \frametitle{Fazit}

  Je Anforderungen an das Webframework (“Taste”)\\
  \dots
\end{frame}


\begin{frame}
  \frametitle{Quellen der Abbildungen}
  \footnotesize
  \begin{enumerate}[<1->]
    \item Innenhof Informatik
        \url{http://www.flickr.com/photos/bennybenny/3597853896/} \label{illu:1}
  \end{enumerate}
  alle URLs aufgerufen am 14. November 2012.
\end{frame}


\begin{frame}
  \frametitle{Quellen}
http://www.infoworld.com/d/application-development/pillars-python-six-python-web-frameworks-compared-169442
http://wiki.python.org/moin/WebFrameworks
http://wiki.python-forum.de/Web-Frameworks
http://blog.ianbicking.org/turbogears-and-pylons.html
\end{frame}


\begin{frame}
    \frametitle{Ende der Präsentation}
    \LARGE
    \begin{itemize}[<1->]
        \item Vielen Dank für Ihre Aufmerksamkeit.
        \item
        \item offene Fragen?
        \item Diskussion
        \begin{itemize}[<1->]
          \Large
          \item Kritik, Anregungen
        \end{itemize}
    \end{itemize}
\end{frame}

\end{document}
